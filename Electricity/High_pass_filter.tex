% The code is quite messy !!!

\documentclass[tikz,border=10pt]{standalone}

% Useful packages
\usepackage{amsmath}
\usepackage[european, straightvoltages, RPvoltages, cute inductor]{circuitikz}  % RPvoltages definie la convention des dipoles (sens tension faux sinon)

\begin{document}

\begin{circuitikz}        
    % Circuit code
    \draw (0,0) to[short,o-o] ++ (4,0);
    \draw (0,2) to[C, l_=C, o-] ++ (3,0) coordinate(a);
    \draw (a) to[short, -o] ++ (1,0);
    \draw (a) to[R, name=R, *-*] ++(0,-2);
    \node at (R.center) {R};  % draw label "R" at the center of the resistance

    % Voltage labels
    \draw (0,2) to[open,v=V$_{\text{in}}$\;] ++(0,-2);
    \draw (4,2) to[open,v^=\hspace{1.5mm} V$_{\text{out}}$] ++(0,-2);
\end{circuitikz}

\vspace{5cm}

\begin{circuitikz}        
    % Circuit code
    \draw (0,0) to[short,o-o] ++ (4,0);  % draw the bottom wire
    \draw (0,2) to[R, name=R, o-] ++ (3,0) coordinate(a);  % draw the resistance
    \node at (R.center) {R};  % draw label "R" at the center of the resistance
    \draw (a) to[short,-o] ++ (1,0);  % draw wire to the right of R
    \draw (a) to[L, l_=L, *-*] ++(0,-2);  % draw the Capacitor

    % Voltage labels
    \draw (0,2) to[open,v_=V$_{\text{in}}$\;] ++(0,-2);
    \draw (4,2) to[open,v^=\hspace{1.5mm} V$_{\text{out}}$] ++(0,-2);
\end{circuitikz}

\end{document}